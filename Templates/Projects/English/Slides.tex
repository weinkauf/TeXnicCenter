%description: Basic Slides in English
%% Based on a TeXnicCenter-Template by Gyorgy SZEIDL.
%%%%%%%%%%%%%%%%%%%%%%%%%%%%%%%%%%%%%%%%%%%%%%%%%%%%%%%%%%%%%

%----------------------------------------------------------
%
\documentclass[titlepage,leqno]{slides}%
%
%----------------------------------------------------------
% This is a sample document for the LaTeX Slides Class
% Class options
%       --  Body text point size (normalsize) is 27 (default)
%           and can not be adjusted to any other value.
%       --  Paper size:  letterpaper (8.5x11 inch, default)
%                        a4paper, a5paper, b5paper,
%                        legalpaper, executivepaper
%       --  Orientation (portrait is the default):
%                        landscape
%       --  Quality:     final(default), draft
%       --  Title page:  titlepage, notitlepage
%       --  Columns:     onecolumn (default), [twocolumn is not avalible]
%       --  Equation numbering (equation numbers on the right is the default)
%                        leqno (equation numbers on the left)
%       --  Displayed equations (centered is the default)
%                    fleqn (flush left)
%
%  \documentclass[a4paper,fleqn]{slides}
%
%  The slides are separated from each other by the slide
%  environment, see below:
%
\usepackage{amsmath}%
\usepackage{amsfonts}%
\usepackage{amssymb}%
\usepackage{graphicx}
%------------------------------------------------------------------
\hfuzz5pt % Don't bother to report overfull boxes < 5pt
\newtheorem{theorem}{Theorem}
\newtheorem{corollary}[theorem]{Corollary}
\newtheorem{definition}[theorem]{Definition}
\newtheorem{example}[theorem]{Example}
\newtheorem{exercise}[theorem]{Exercise}
\newtheorem{lemma}[theorem]{Lemma}
\newtheorem{proposition}[theorem]{Proposition}
\pagestyle{plain}
%%% --------------------------------------------------------------
\begin{document}
\title{This is the Title}
\author{Author}
\date{The date}
\maketitle
% ----------------------------------------------------------------
\begin{slide}{1}

This is the first slide.

You can typeset \emph{Emphasized text}.

You can also typeset  \textbf{Bold}, \textit{Italics},
\textsl{Slanted} and \texttt{Typewriter} text. Roman fonts are not
available.

Point size can be changed by making use of the {\tiny tiny},
{\scriptsize scriptsize}, {\footnotesize footnotesize}, {\small
small}, {\normalsize normalsize}, {\large large}, {\Large Large},
{\LARGE LARGE}, {\huge huge} and {\Huge Huge} commands.

\end{slide}
% ----------------------------------------------------------------
\begin{slide}{2}
This is the second slide.

The numbered equation
\begin{equation}
u_{tt}-\Delta u+u^{5}+u\left|  u\right|  ^{p-2}=0\text{ in }\mathbf{R}%
^{3}\times\left[  0,\infty\right[  .\label{eqn1}%
\end{equation}
is automatically numbered as equation \ref{eqn1}.

\end{slide}
% ----------------------------------------------------------------

\end{document}
