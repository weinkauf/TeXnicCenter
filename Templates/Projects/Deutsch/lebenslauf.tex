%description: Lebenslaufvorlage unter Verwendung von currvita.sty
%% Basierend auf einer TeXnicCenter-Vorlage von Mark M�ller
%%%%%%%%%%%%%%%%%%%%%%%%%%%%%%%%%%%%%%%%%%%%%%%%%%%%%%%%%%%%%%%%%%%%%%%

% W�hlen Sie die Optionen aus, indem Sie % vor der Option entfernen  
% Dokumentation von currvita.sty: currvita.tex

%%%%%%%%%%%%%%%%%%%%%%%%%%%%%%%%%%%%%%%%%%%%%%%%%%%%%%%%%%%%%%%%%%%%%%%
%% Optionen zum Layout des Curriculum Vitae                          %%
%%%%%%%%%%%%%%%%%%%%%%%%%%%%%%%%%%%%%%%%%%%%%%%%%%%%%%%%%%%%%%%%%%%%%%%
\documentclass[12pt, a4paper]{article}
\usepackage[%
%LabelsAligned,					% engere vertikale Zwischenr�ume
%NoDate									% keine Datumsangabe am Ende
]{currvita}


%% Deutsche Anpassungen %%%%%%%%%%%%%%%%%%%%%%%%%%%%%%%%%%%%%
\usepackage[ngerman]{babel}
\usepackage[T1]{fontenc}
\usepackage[ansinew]{inputenc}

\usepackage{lmodern} %Type1-Schriftart f�r nicht-englische Texte


%% Packages f�r Grafiken & Abbildungen %%%%%%%%%%%%%%%%%%%%%%
\usepackage{graphicx} %%Zum Laden von Grafiken
%\usepackage{subfig} %%Teilabbildungen in einer Abbildung
%\usepackage{pst-all} %%PSTricks - nicht verwendbar mit pdfLaTeX
\usepackage{eso-pic} %%F�r das Bewerbungsbild

%% Beachten Sie:
%% Die Einbindung einer Grafik erfolgt mit \includegraphics{Dateiname}
%% bzw. �ber den Dialog im Einf�gen-Men�.
%% 
%% Im Modus "LaTeX => PDF" k�nnen Sie u.a. folgende Grafikformate verwenden:
%%   .jpg  .png  .pdf  .mps
%% 
%% In den Modi "LaTeX => DVI", "LaTeX => PS" und "LaTeX => PS => PDF"
%% k�nnen Sie u.a. folgende Grafikformate verwenden:
%%   .eps  .ps  .bmp  .pict  .pntg


\begin{document}

%%%%%%%%%%%%%%%%%%%%%%%%%%%%%%%%%%%%%%%%%%%%%%%%%%%%%%%%%%%%%%%%%%%%%%
%% Ihr Bewerbungsbild direkt auf dem Lebenslauf?                    %%
%%%%%%%%%%%%%%%%%%%%%%%%%%%%%%%%%%%%%%%%%%%%%%%%%%%%%%%%%%%%%%%%%%%%%%
%\AddToShipoutPicture*{%
%  \put(450,610){\includegraphics[width=39mm]{Ihr Bild}}%
%}

%%%%%%%%%%%%%%%%%%%%%%%%%%%%%%%%%%%%%%%%%%%%%%%%%%%%%%%%%%%%%%%%%%%%%%
%% Ihr Lebenslauf                                                   %%
%%%%%%%%%%%%%%%%%%%%%%%%%%%%%%%%%%%%%%%%%%%%%%%%%%%%%%%%%%%%%%%%%%%%%%

\begin{cv}{Lebenslauf}	% �berschrift des Lebenslaufs frei w�hlbar
%%%%%%%%%%%%%%%%%%%%%%%%%%%%%%%%%%%%%%%%%%%%%%%%%%%%%%%%%%%%%%%%%%%%%%
%% Einzelne Abschnitte des CV kommen in einzelne Umgebungen der Form:%
%% \begin{cvlist}{�berschrift}                                       %
%% \item                                                             %
%% ...             																									 %
%% \end{cvlist}																											 %
%% Zu unterscheiden sind \item mit Argument (\item[Argument] Inhalt) %
%% und ohne (\item Inhalt). Folgend ein Anwendungsbeispiel. Der Autor% 
%% des Styles legt seine eigene ausf�hrliche CV als cvtest.tex der   %
%% Dokumentation bei.																								 %
%%%%%%%%%%%%%%%%%%%%%%%%%%%%%%%%%%%%%%%%%%%%%%%%%%%%%%%%%%%%%%%%%%%%%%

%% Abschnitt 1 %%%%%%%%%%%%%%%%%%%%%%%%%%%%%%%%%%%%%%%%%%%%%%%%%%%%%%%
\begin{cvlist}{Personalien}	% �berschrift des Abschnitts
\item Vorname Nachname\\		% die \item haben untereinander einen 
 			Stra�e Nr.\\					% gr��eren Abstand, als eine neue Zeile
 			Postleitzahl Ort
\item Tel.: (030) 12\,34\,56\\
			email: irgenwer@irgendwo.ch
\item Geb.: 30. Februar 2003, Ort
\item mehrfach geschieden, 5 Kinder
\end{cvlist}
%%%%%%%%%%%%%%%%%%%%%%%%%%%%%%%%%%%%%%%%%%%%%%%%%%%%%%%%%%%%%%%%%%%%%%

%% Abschnitt 2 %%%%%%%%%%%%%%%%%%%%%%%%%%%%%%%%%%%%%%%%%%%%%%%%%%%%%%%
\begin{cvlist}{Schulbildung}
\item[09/82--06/85] Grundschule
\item[07/85--12/92] F�rderschule
\item[01/93--07/97] Gymnasium, Abschluss Abitur
\end{cvlist}
%%%%%%%%%%%%%%%%%%%%%%%%%%%%%%%%%%%%%%%%%%%%%%%%%%%%%%%%%%%%%%%%%%%%%%

%% Abschnitt 3 %%%%%%%%%%%%%%%%%%%%%%%%%%%%%%%%%%%%%%%%%%%%%%%%%%%%%%%
\begin{cvlist}{Studium}
\item[04/99--07/99] Soziologiestudium, FU-Berlin
\item[04/00--09/02] Theatherp�dagogik, Uni-Z�rich
\item[10/02--07/04] Medizinstudium, Uni-Ludwigshafen, Abschluss 1,0
\end{cvlist}
%%%%%%%%%%%%%%%%%%%%%%%%%%%%%%%%%%%%%%%%%%%%%%%%%%%%%%%%%%%%%%%%%%%%%%

%% Abschnitt 4 %%%%%%%%%%%%%%%%%%%%%%%%%%%%%%%%%%%%%%%%%%%%%%%%%%%%%%%
% Beachten Sie die Wirkung kurzer �berschriften und Items ohne Argument!
\begin{cvlist}{Sprachen}
\item Japanisch, Norwegisch, Polnisch, Deutsch, Perl, C++
\end{cvlist}
%%%%%%%%%%%%%%%%%%%%%%%%%%%%%%%%%%%%%%%%%%%%%%%%%%%%%%%%%%%%%%%%%%%%%%

%% Abschnitt 5 %%%%%%%%%%%%%%%%%%%%%%%%%%%%%%%%%%%%%%%%%%%%%%%%%%%%%%%
\begin{cvlist}{Interessen}
\item[Musik] h�ren und dr�ber reden
\item[Layout] \LaTeX -Fetischist
\end{cvlist}
%%%%%%%%%%%%%%%%%%%%%%%%%%%%%%%%%%%%%%%%%%%%%%%%%%%%%%%%%%%%%%%%%%%%%%

%% Angaben der Abschlusszeile %%%%%%%%%%%%%%%%%%%%%%%%%%%%%%%%%%%%%%%%
\cvplace{Ort}						% Erstellungsort
%\date{}								% falls anderes, als das aktuelle gew�nschte

\end{cv}
\end{document}
