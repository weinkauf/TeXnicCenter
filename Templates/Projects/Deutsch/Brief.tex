%description: Standardbrief in Deutsch
%%%%%%%%%%%%%%%%%%%%%%%%%%%%%%%%%%%%%%%%%%%%%%%%%%%%%%%%%%%%%
%% HEADER
%%%%%%%%%%%%%%%%%%%%%%%%%%%%%%%%%%%%%%%%%%%%%%%%%%%%%%%%%%%%%
\documentclass[addressstd,a4paper,10pt]{dinbrief}
% Alternative Settings:
%		a4paper / letter
%   addressstd / addresshigh


%% Normales LaTeX oder pdfLaTeX? %%%%%%%%%%%%%%%%%%%%%%%%%%%%
%% ==> Das neue if-Kommando "\ifpdf" wird an einigen wenigen
%% ==> Stellen ben�tigt, um die Kompatibilit�t zwischen
%% ==> LaTeX und pdfLaTeX herzustellen.
\newif\ifpdf
\ifx\pdfoutput\undefined
	\pdffalse              %%normales LaTeX wird ausgef�hrt
\else
	\pdfoutput=1           
	\pdftrue               %%pdfLaTeX wird ausgef�hrt
\fi


%% Fonts f�r pdfLaTeX %%%%%%%%%%%%%%%%%%%%%%%%%%%%%%%%%%%%%%%
%% ==> Nur notwendig, falls keine cm-super-Fonts installiert
\ifpdf
	%\usepackage{ae}       %%Benutzen Sie nur eines dieser Pakete:
	%\usepackage{zefonts}  %%je nachdem, welches Sie besitzen.
\else
	%%Normales LaTeX - keine speziellen Fontpackages notwendig
\fi


%% Deutsche Anpassungen %%%%%%%%%%%%%%%%%%%%%%%%%%%%%%%%%%%%%
\usepackage[ngerman]{babel}
\usepackage[T1]{fontenc}
\usepackage[latin1]{inputenc}

%% Packages f�r Grafiken %%%%%%%%%%%%%%%%%%%%%%%%%%%%%%%%%%%%
\ifpdf %%Einbindung von Grafiken mittels \includegraphics{datei}
	\usepackage[pdftex]{graphicx} %%Grafiken in pdfLaTeX
\else
	\usepackage[dvips]{graphicx} %%Grafiken und normales LaTeX
\fi


%% Kommandos zur Anpassung %%%%%%%%%%%%%%%%%%%%%%%%%%%%%%%%%%
%\renewcommand{\ccname}{\tt Verteiler}
%%\renewcommand{\enclname}{\tt Anlage(n)}
%\renewcommand{\psname}{PS}
%\renewcommand{\phonemsg}{\tt Telefon}
%\renewcommand{\signmsgold}{\tt Unsere Zeichen}
%\renewcommand{\signmsgnew}{\tt Unsere Zeichen, unsere Nachricht vom}
%\renewcommand{\yourmailmsg}{\tt Ihre Zeichen, Ihre Nachricht vom}


%% Absenderaddresse %%%%%%%%%%%%%%%%%%%%%%%%%%%%%%%%%%%%%%%%
\address{Vorname Nachname\\
Stra�e Nr. 1\\
11111 Stadt}

\place{Stadt}
\phone{030}{11111111}
\signature{Vorname Nachname}

%\nobackaddressrule
\backaddress{Vorname Nachname, Stra�e Nr. 1, 11111 Stadt}


%% Text am Ende des Briefes %%%%%%%%%%%%%%%%%%%%%%%%%%%%%%%%%
\bottomtext{%
  \makebox[\textwidth][c]{\small\scshape
   Bankverbindung $\cdot$ Web Addresse $\cdot$ Andere Informationen
  }
}


%% Datum %%%%%%%%%%%%%%%%%%%%%%%%%%%%%%%%%%%%%%%%%%%%%%%%%%%%
%\date{\today} %%Wenn kommentiert, wird das aktuelle Datum verwendet


%% Etiketten %%%%%%%%%%%%%%%%%%%%%%%%%%%%%%%%%%%%%%%%%%%%%%%%
%\labelstyle{plain} %%Art der Etiketten: nur plain m�glich
%\makelabels %%Generiert Etiketten


%% Styles %%%%%%%%%%%%%%%%%%%%%%%%%%%%%%%%%%%%%%%%%%%%%%%%%%%
%\nowindowrules %%Keinen Rahmen um die Empf�ngeraddresse
%\nowindowtics %%Keine Faltkennzeichnungen
\centeraddress %%Vertikale Zentrierung der Empf�ngeraddresse
\pagestyle{empty} %%Keine Kopf- oder Fu�zeilen



%%%%%%%%%%%%%%%%%%%%%%%%%%%%%%%%%%%%%%%%%%%%%%%%%%%%%%%%%%%%%
%% DOCUMENT - BRIEF(E)
%%%%%%%%%%%%%%%%%%%%%%%%%%%%%%%%%%%%%%%%%%%%%%%%%%%%%%%%%%%%%
\begin{document}

%% Dateiendungen f�r Grafiken %%%%%%%%%%%%%%%%%%%%%%%%%%%%%%%
%% ==> Sie k�nnen hiermit die Dateiendung einer Grafik weglassen.
%% ==> Aus "\includegraphics{titel.eps}" wird "\includegraphics{titel}".
%% ==> Wenn Sie nunmehr 2 inhaltsgleiche Grafiken "titel.eps" und
%% ==> "titel.pdf" erstellen, wird jeweils nur die Grafik eingebunden,
%% ==> die von ihrem Compiler verarbeitet werden kann.
%% ==> pdfLaTeX benutzt "titel.pdf". LaTeX benutzt "titel.eps".
\ifpdf
	\DeclareGraphicsExtensions{.pdf,.jpg,.png}
\else
	\DeclareGraphicsExtensions{.eps}
\fi


%% Beginn des Briefes %%%%%%%%%%%%%%%%%%%%%%%%%%%%%%%%%%%%%%%
\begin{letter}{Vorname Name\\
               Stra�e Nr. 2\\[\medskipamount]
               11111 Stadt}

\yourmail{A} %%Ihre Zeichen
\sign{B} %%Unsere Zeichen

%\writer{} %%Name des Sachbearbeiters

\subject{\bf Betreffzeile}

\opening{Sehr geehrter Herr Name,}

Ich schreibe Ihnen einen Brief. Ich schreibe Ihnen einen Brief. Ich schreibe Ihnen einen Brief. Ich schreibe Ihnen einen Brief. Ich schreibe Ihnen einen Brief. Ich schreibe Ihnen einen Brief. Ich schreibe Ihnen einen Brief. Ich schreibe Ihnen einen Brief. Ich schreibe Ihnen einen Brief. Ich schreibe Ihnen einen Brief. Ich schreibe Ihnen einen Brief. Ich schreibe Ihnen einen Brief.

Ich schreibe Ihnen einen Brief. Ich schreibe Ihnen einen Brief. Ich schreibe Ihnen einen Brief. Ich schreibe Ihnen einen Brief. Ich schreibe Ihnen einen Brief. Ich schreibe Ihnen einen Brief. Ich schreibe Ihnen einen Brief. Ich schreibe Ihnen einen Brief. Ich schreibe Ihnen einen Brief. Ich schreibe Ihnen einen Brief. Ich schreibe Ihnen einen Brief. Ich schreibe Ihnen einen Brief. Ich schreibe Ihnen einen Brief. Ich schreibe Ihnen einen Brief. Ich schreibe Ihnen einen Brief. Ich schreibe Ihnen einen Brief.

\closing{Mit freundlichen Gr��en}

%\ps{PostScriptum. Was auch immer.} %%PostScriptum

%\encl[Anlage(n)]{Text f�r Anlagen} %%Anlagen

\end{letter}

%% Ende des Briefes %%%%%%%%%%%%%%%%%%%%%%%%%%%%%%%%%%%%%%%%%


%% Beginn des zweiten Briefes %%%%%%%%%%%%%%%%%%%%%%%%%%%%%%%
%\begin{letter}{Vorname Name...
%%
%% Sie k�nnen hier einen weiteren Brief beginnen.
%% Kopieren Sie dazu die Kommandos des ersten Briefes an
%% diese Stelle und passen Sie diese entsprechend an.
%%
%%\end{letter}
%% Ende des zweiten Briefes %%%%%%%%%%%%%%%%%%%%%%%%%%%%%%%%%


\end{document}
