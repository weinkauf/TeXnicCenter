%description: Standardbrief in Deutsch
%% Basierend auf einer TeXnicCenter-Vorlage von Tino Weinkauf.
%%%%%%%%%%%%%%%%%%%%%%%%%%%%%%%%%%%%%%%%%%%%%%%%%%%%%%%%%%%%%%

%%%%%%%%%%%%%%%%%%%%%%%%%%%%%%%%%%%%%%%%%%%%%%%%%%%%%%%%%%%%%
%% HEADER
%%%%%%%%%%%%%%%%%%%%%%%%%%%%%%%%%%%%%%%%%%%%%%%%%%%%%%%%%%%%%
\documentclass[addressstd,a4paper,10pt]{dinbrief}
% Alternative Settings:
%		a4paper / letter
%   addressstd / addresshigh


%% Deutsche Anpassungen %%%%%%%%%%%%%%%%%%%%%%%%%%%%%%%%%%%%%
\usepackage[ngerman]{babel}
\usepackage[T1]{fontenc}
\usepackage[ansinew]{inputenc}

\usepackage{lmodern} %Type1-Schriftart f�r nicht-englische Texte


%% Packages f�r Grafiken & Abbildungen %%%%%%%%%%%%%%%%%%%%%%
\usepackage{graphicx} %%Zum Laden von Grafiken
%\usepackage{subfig} %%Teilabbildungen in einer Abbildung
%\usepackage{tikz} %%Vektorgrafiken aus LaTeX heraus erstellen

%% Beachten Sie:
%% Die Einbindung einer Grafik erfolgt mit \includegraphics{Dateiname}
%% bzw. �ber den Dialog im Einf�gen-Men�.
%% 
%% Im Modus "LaTeX => PDF" k�nnen Sie u.a. folgende Grafikformate verwenden:
%%   .jpg  .png  .pdf  .mps
%% 
%% In den Modi "LaTeX => DVI", "LaTeX => PS" und "LaTeX => PS => PDF"
%% k�nnen Sie u.a. folgende Grafikformate verwenden:
%%   .eps  .ps  .bmp  .pict  .pntg


%% Kommandos zur Anpassung %%%%%%%%%%%%%%%%%%%%%%%%%%%%%%%%%%
%\renewcommand{\ccname}{\tt Verteiler}
%%\renewcommand{\enclname}{\tt Anlage(n)}
%\renewcommand{\psname}{PS}
%\renewcommand{\phonemsg}{\tt Telefon}
%\renewcommand{\signmsgold}{\tt Unsere Zeichen}
%\renewcommand{\signmsgnew}{\tt Unsere Zeichen, unsere Nachricht vom}
%\renewcommand{\yourmailmsg}{\tt Ihre Zeichen, Ihre Nachricht vom}


%% Absenderaddresse %%%%%%%%%%%%%%%%%%%%%%%%%%%%%%%%%%%%%%%%
\address{Vorname Nachname\\
Stra�e Nr. 1\\
11111 Stadt}

\place{Stadt}
\phone{030}{11111111}
\signature{Vorname Nachname}

%\nobackaddressrule
\backaddress{Vorname Nachname, Stra�e Nr. 1, 11111 Stadt}


%% Text am Ende des Briefes %%%%%%%%%%%%%%%%%%%%%%%%%%%%%%%%%
\bottomtext{%
  \makebox[\textwidth][c]{\small\scshape
   Bankverbindung $\cdot$ Web Addresse $\cdot$ Andere Informationen
  }
}


%% Datum %%%%%%%%%%%%%%%%%%%%%%%%%%%%%%%%%%%%%%%%%%%%%%%%%%%%
%\date{\today} %%Wenn kommentiert, wird das aktuelle Datum verwendet


%% Etiketten %%%%%%%%%%%%%%%%%%%%%%%%%%%%%%%%%%%%%%%%%%%%%%%%
%\labelstyle{plain} %%Art der Etiketten: nur plain m�glich
%\makelabels %%Generiert Etiketten


%% Styles %%%%%%%%%%%%%%%%%%%%%%%%%%%%%%%%%%%%%%%%%%%%%%%%%%%
%\nowindowrules %%Keinen Rahmen um die Empf�ngeraddresse
%\nowindowtics %%Keine Faltkennzeichnungen
\centeraddress %%Vertikale Zentrierung der Empf�ngeraddresse
\pagestyle{empty} %%Keine Kopf- oder Fu�zeilen



%%%%%%%%%%%%%%%%%%%%%%%%%%%%%%%%%%%%%%%%%%%%%%%%%%%%%%%%%%%%%
%% DOCUMENT - BRIEF(E)
%%%%%%%%%%%%%%%%%%%%%%%%%%%%%%%%%%%%%%%%%%%%%%%%%%%%%%%%%%%%%
\begin{document}

%% Beginn des Briefes %%%%%%%%%%%%%%%%%%%%%%%%%%%%%%%%%%%%%%%
\begin{letter}{Vorname Name\\
               Stra�e Nr. 2\\[\medskipamount]
               11111 Stadt}

\yourmail{A} %%Ihre Zeichen
\sign{B} %%Unsere Zeichen

%\writer{} %%Name des Sachbearbeiters

\subject{\bf Betreffzeile}

\opening{Sehr geehrter Herr Name,}

Ich schreibe Ihnen einen Brief. Ich schreibe Ihnen einen Brief. Ich schreibe Ihnen einen Brief. Ich schreibe Ihnen einen Brief. Ich schreibe Ihnen einen Brief. Ich schreibe Ihnen einen Brief. Ich schreibe Ihnen einen Brief. Ich schreibe Ihnen einen Brief. Ich schreibe Ihnen einen Brief. Ich schreibe Ihnen einen Brief. Ich schreibe Ihnen einen Brief. Ich schreibe Ihnen einen Brief.

Ich schreibe Ihnen einen Brief. Ich schreibe Ihnen einen Brief. Ich schreibe Ihnen einen Brief. Ich schreibe Ihnen einen Brief. Ich schreibe Ihnen einen Brief. Ich schreibe Ihnen einen Brief. Ich schreibe Ihnen einen Brief. Ich schreibe Ihnen einen Brief. Ich schreibe Ihnen einen Brief. Ich schreibe Ihnen einen Brief. Ich schreibe Ihnen einen Brief. Ich schreibe Ihnen einen Brief. Ich schreibe Ihnen einen Brief. Ich schreibe Ihnen einen Brief. Ich schreibe Ihnen einen Brief. Ich schreibe Ihnen einen Brief.

\closing{Mit freundlichen Gr��en}

%\ps{PostScriptum. Was auch immer.} %%PostScriptum

%\encl[Anlage(n)]{Text f�r Anlagen} %%Anlagen

\end{letter}

%% Ende des Briefes %%%%%%%%%%%%%%%%%%%%%%%%%%%%%%%%%%%%%%%%%


%% Beginn des zweiten Briefes %%%%%%%%%%%%%%%%%%%%%%%%%%%%%%%
%\begin{letter}{Vorname Name...
%%
%% Sie k�nnen hier einen weiteren Brief beginnen.
%% Kopieren Sie dazu die Kommandos des ersten Briefes an
%% diese Stelle und passen Sie diese entsprechend an.
%%
%%\end{letter}
%% Ende des zweiten Briefes %%%%%%%%%%%%%%%%%%%%%%%%%%%%%%%%%


\end{document}
