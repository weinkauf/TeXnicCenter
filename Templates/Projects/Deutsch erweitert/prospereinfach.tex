%description: Folienvorlage unter Verwendung der Prosperklasse
%% Basierend auf einer TeXnicCenter-Vorlage von Mark M�ller
%%%%%%%%%%%%%%%%%%%%%%%%%%%%%%%%%%%%%%%%%%%%%%%%%%%%%%%%%%%%%%%%%%%%%%%

% Nutzen Sie zum Kompilieren LaTeX => PS, als Anzeigeprogramm GhostView
% W�hlen Sie die Optionen aus, indem Sie % vor der Option entfernen  
% Dokumtentation von Prosper: prosper-doc.pdf

%%%%%%%%%%%%%%%%%%%%%%%%%%%%%%%%%%%%%%%%%%%%%%%%%%%%%%%%%%%%%%%%%%%%%%%
%% Optionen zum Layout der Folien                                    %%
%%%%%%%%%%%%%%%%%%%%%%%%%%%%%%%%%%%%%%%%%%%%%%%%%%%%%%%%%%%%%%%%%%%%%%%

\documentclass[%
%nontotal,					% nur aktuelle Seitennummer
%slideBW,						% f�r schwarz/wei� Druck
%nocolorBG,					% Hintergrund ist durchsichtig 
ps,									% nur Postscript (f�r Folien)
%noFooter,					% keine Angabe im Fu� der Folie\mid\mid
%frames,						% vordefinierte Folien- und Pr�-
%azure,							% sentationsvorlagen
%contemporain,
%nuancegris,
%troispoints,
%lignesbleues,
%darkblue,
%alienglow,
%autumn
]{prosper}

%% Deutsche Anpassungen %%%%%%%%%%%%%%%%%%%%%%%%%%%%%%%%%%%%%
\usepackage[ngerman]{babel}
\usepackage[ansinew]{inputenc}

%% Falls die automatische Worttrennung in W�rtern mit Umlauten
%% nicht funktionieren sollte oder der Text pixelig aussieht:
%% ==> Installieren Sie die cm-super Fonts (z.B. mit dem mikTeX Package Manager).
%% Eine nicht ganz vollwertige Alternative ist die Verwendung dieses Pakets:
%\usepackage{ae, aeguill}

%% Titelangaben der Folien %%%%%%%%%%%%%%%%%%%%%%%%%%%%%%%%%%%%%%%%%%%
\title{Titel der Pr�sentation}
\subtitle{Prosper -- ein M�glichkeit zum Erzeugen von Folien}
\author{Ihr Name}
\institution{Name Ihres Arbeitgebers}
\email{irgendwer@irgendwo.at}
\slideCaption{alternativer Titel f�r den Fu� jeder Seite}

\begin{document}
\maketitle					% Titelfolie wird erzeugt

%% Folie %%%%%%%%%%%%%%%%%%%%%%%%%%%%%%%%%%%%%%%%%%%%%%%%%%%%%%%%%%%%%
\begin{slide}{�berschrift der Folie}
\begin{itemize}
	\item Der Platz f�r den ersten Stichpunkt
	\item Und ein anderer Stichpunkt
	\item Oder man untergliedert:
		\begin{itemize}
			\item ein Unterpunkt, 
			\item und noch einer.
		\end{itemize}
	\item Machen wir noch einen -- dann ist aber Schluss!
\end{itemize}
\end{slide}
%%%%%%%%%%%%%%%%%%%%%%%%%%%%%%%%%%%%%%%%%%%%%%%%%%%%%%%%%%%%%%%%%%%%%%

%% Folie %%%%%%%%%%%%%%%%%%%%%%%%%%%%%%%%%%%%%%%%%%%%%%%%%%%%%%%%%%%%%
\begin{slide}{Der Punkt}
\begin{itemize}
		\item Das Aussehen des Punkts kann �brigens mit \verb#\myitem{Nummer der Gliederungsebene}{Zeichen}# definiert werden.
		\item \verb#\myitem{2}{$\circ$}# ver�ndert z.\,B. in der zweiten Gliederungsebene den gr�nen 3D-Punkt in einen schlichten $\circ$.
		\item Das muss man vor \verb#\begin{document}# definieren.
\end{itemize}
\end{slide}
%%%%%%%%%%%%%%%%%%%%%%%%%%%%%%%%%%%%%%%%%%%%%%%%%%%%%%%%%%%%%%%%%%%%%%

%% Folie %%%%%%%%%%%%%%%%%%%%%%%%%%%%%%%%%%%%%%%%%%%%%%%%%%%%%%%%%%%%%
\begin{slide}{Positionierung }
\vspace{2cm}
\begin{itemize}
	\item Zur vertikalen Positionierung schlage ich vor, \verb#\vspace{Ma�}# 
				zu verwenden
	\item In diesem Beispiel sind es �brigens 2 cm
\end{itemize}
\end{slide}
%%%%%%%%%%%%%%%%%%%%%%%%%%%%%%%%%%%%%%%%%%%%%%%%%%%%%%%%%%%%%%%%%%%%%%

%% Folie %%%%%%%%%%%%%%%%%%%%%%%%%%%%%%%%%%%%%%%%%%%%%%%%%%%%%%%%%%%%%
\begin{slide}{M�glichkeiten}
\begin{itemize}
	\item Ich nutze hier \verb#Prosper# nur rudiment�r.
	\item Es kann zum Erstellen von Pr�sentationen mit �berblendungen verwendet werden.
	\item Dazu verwendet man als Anzeige z.\,B. den \verb#Acrobat Reader# von \verb#Adobe#.
	\item Ein Beispiel f�r die Funktionalit�t von \verb#Prosper# ist \verb#prosper-tour.pdf# -- zu finden im Dokumentationsordner von \verb#Prosper#.
\end{itemize}
\end{slide}
%%%%%%%%%%%%%%%%%%%%%%%%%%%%%%%%%%%%%%%%%%%%%%%%%%%%%%%%%%%%%%%%%%%%%%

\end{document}