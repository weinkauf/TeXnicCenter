%description: Zus�tzliche Dokumentoptionen (Teilabbildungen u.a.)
%% Basierend auf einer TeXnicCenter-Vorlage von Tino Weinkauf.
%%%%%%%%%%%%%%%%%%%%%%%%%%%%%%%%%%%%%%%%%%%%%%%%%%%%%%%%%%%%%%

%%%%%%%%%%%%%%%%%%%%%%%%%%%%%%%%%%%%%%%%%%%%%%%%%%%%%%%%%%%%%
%% OPTIONEN
%%%%%%%%%%%%%%%%%%%%%%%%%%%%%%%%%%%%%%%%%%%%%%%%%%%%%%%%%%%%%
%%
%% ACHTUNG: Sie ben�tigen ein Hauptdokument, um diese Datei
%%          benutzen zu k�nnen. Verwenden Sie im Hauptdokument
%%          den Befehl "\input{dateiname}", um diese
%%          Datei einzubinden.
%%


%%%%%%%%%%%%%%%%%%%%%%%%%%%%%%%%%%%%%%%%%%%%%%%%%%%%%%%%%%%%%
%% OPTIONEN F�R SUBFIGURES - TEILABBILDUNGEN
%%%%%%%%%%%%%%%%%%%%%%%%%%%%%%%%%%%%%%%%%%%%%%%%%%%%%%%%%%%%%
%%Neudefinition der Labelung f�r Subfigures
%%Beispiel:
%%In der Abbildung 4.1 schlicht: (a)
%%Als Referenz im Text: 4.1a
%\makeatletter
%    \renewcommand{\thesubfigure}{\alph{subfigure}}
%    \renewcommand{\@thesubfigure}{\subcaplabelfont (\thesubfigure)\space}
%    \renewcommand{\p@subfigure}{\thefigure}
%\makeatother

%%Definitionen f�r Abst�nde zwischen den Subfigures
%% Die Kommandos '\goodgap' und '\littlegap' werden zur Verf�gung gestellt.
%\newlength{\lengthgoodgap}
%\addtolength{\lengthgoodgap}{\subfigtopskip}
%\addtolength{\lengthgoodgap}{\subfigbottomskip}
%\newcommand{\goodgap}{\hspace{\lengthgoodgap}}
%\newlength{\lengthlittlegap}
%\addtolength{\lengthlittlegap}{\subfigtopskip}
%\newcommand{\littlegap}{\hspace{\lengthlittlegap}}

%%Definition f�r Gr��e von Bildern, wenn 2 in einer Zeile sind
%\newlength{\twopicwidth}
%\addtolength{\twopicwidth}{0.5\linewidth}
%\addtolength{\twopicwidth}{-0.5\lengthgoodgap}

%%Definition f�r Gr��e von Bildern, wenn 3 in einer Zeile sind
%\newlength{\threepicwidth}
%\addtolength{\threepicwidth}{0.333333333\linewidth}
%\addtolength{\threepicwidth}{-0.666666666\lengthlittlegap}


%%Hier ein Beispiel f�r die Anwendung obiger Definitionen:
%% ==> Benutzen Sie dies als Vorlage im Text; nicht hier.
% \begin{figure}%
% \subfigure[SubName1]%
% {\label{SubLabel1}\includegraphics[width=\twopicwidth]{bild1}}%
% \goodgap%
% \subfigure[SubName2]%
% {\label{SubLabel2}\includegraphics[width=\twopicwidth]{bild2}}%
% \\%New Line
% \subfigure[SubName3]%
% {\label{SubLabel3}\includegraphics[width=\twopicwidth]{bild3}}%
% \goodgap%
% \subfigure[SubName4]%
% {\label{SubLabel4}\includegraphics[width=\twopicwidth]{bild4}}%
% \caption[Kurze Bezeichnung der Abbildung f�r das Abbildungsverzeichnis]%
% {L�ngere Bezeichnung der Abbildung, die direkt unter der Abbildung erscheint.}%
% \label{LabelFuerDieGesamteAbbildung}%
% \end{figure}


%%%%%%%%%%%%%%%%%%%%%%%%%%%%%%%%%%%%%%%%%%%%%%%%%%%%%%%%%%%%%
%% ANDERE OPTIONEN UND DEFINITIONEN
%%%%%%%%%%%%%%%%%%%%%%%%%%%%%%%%%%%%%%%%%%%%%%%%%%%%%%%%%%%%%
%%LaTeX soll keine Warnung mehr ausgeben, wenn Overfull Box < 3pt
%\hfuzz3pt

%%Gr��e des Speichers f�r LongTables - h�her gesetzt f�r schnellere Ausf�hrung
%\setcounter{LTchunksize}{400}

